\documentclass[ignorenonframetext,]{beamer}
\setbeamertemplate{caption}[numbered]
\setbeamertemplate{caption label separator}{: }
\setbeamercolor{caption name}{fg=normal text.fg}
\beamertemplatenavigationsymbolsempty
\usepackage{lmodern}
\usepackage{amssymb,amsmath}
\usepackage{ifxetex,ifluatex}
\usepackage{fixltx2e} % provides \textsubscript
\ifnum 0\ifxetex 1\fi\ifluatex 1\fi=0 % if pdftex
\usepackage[T1]{fontenc}
\usepackage[utf8]{inputenc}
\else % if luatex or xelatex
\ifxetex
\usepackage{mathspec}
\else
\usepackage{fontspec}
\fi
\defaultfontfeatures{Ligatures=TeX,Scale=MatchLowercase}
\fi
% use upquote if available, for straight quotes in verbatim environments
\IfFileExists{upquote.sty}{\usepackage{upquote}}{}
% use microtype if available
\IfFileExists{microtype.sty}{%
\usepackage{microtype}
\UseMicrotypeSet[protrusion]{basicmath} % disable protrusion for tt fonts
}{}
\newif\ifbibliography
\usepackage{color}
\usepackage{fancyvrb}
\newcommand{\VerbBar}{|}
\newcommand{\VERB}{\Verb[commandchars=\\\{\}]}
\DefineVerbatimEnvironment{Highlighting}{Verbatim}{commandchars=\\\{\}}
% Add ',fontsize=\small' for more characters per line
\usepackage{framed}
\definecolor{shadecolor}{RGB}{248,248,248}
\newenvironment{Shaded}{\begin{snugshade}}{\end{snugshade}}
\newcommand{\KeywordTok}[1]{\textcolor[rgb]{0.13,0.29,0.53}{\textbf{{#1}}}}
\newcommand{\DataTypeTok}[1]{\textcolor[rgb]{0.13,0.29,0.53}{{#1}}}
\newcommand{\DecValTok}[1]{\textcolor[rgb]{0.00,0.00,0.81}{{#1}}}
\newcommand{\BaseNTok}[1]{\textcolor[rgb]{0.00,0.00,0.81}{{#1}}}
\newcommand{\FloatTok}[1]{\textcolor[rgb]{0.00,0.00,0.81}{{#1}}}
\newcommand{\ConstantTok}[1]{\textcolor[rgb]{0.00,0.00,0.00}{{#1}}}
\newcommand{\CharTok}[1]{\textcolor[rgb]{0.31,0.60,0.02}{{#1}}}
\newcommand{\SpecialCharTok}[1]{\textcolor[rgb]{0.00,0.00,0.00}{{#1}}}
\newcommand{\StringTok}[1]{\textcolor[rgb]{0.31,0.60,0.02}{{#1}}}
\newcommand{\VerbatimStringTok}[1]{\textcolor[rgb]{0.31,0.60,0.02}{{#1}}}
\newcommand{\SpecialStringTok}[1]{\textcolor[rgb]{0.31,0.60,0.02}{{#1}}}
\newcommand{\ImportTok}[1]{{#1}}
\newcommand{\CommentTok}[1]{\textcolor[rgb]{0.56,0.35,0.01}{\textit{{#1}}}}
\newcommand{\DocumentationTok}[1]{\textcolor[rgb]{0.56,0.35,0.01}{\textbf{\textit{{#1}}}}}
\newcommand{\AnnotationTok}[1]{\textcolor[rgb]{0.56,0.35,0.01}{\textbf{\textit{{#1}}}}}
\newcommand{\CommentVarTok}[1]{\textcolor[rgb]{0.56,0.35,0.01}{\textbf{\textit{{#1}}}}}
\newcommand{\OtherTok}[1]{\textcolor[rgb]{0.56,0.35,0.01}{{#1}}}
\newcommand{\FunctionTok}[1]{\textcolor[rgb]{0.00,0.00,0.00}{{#1}}}
\newcommand{\VariableTok}[1]{\textcolor[rgb]{0.00,0.00,0.00}{{#1}}}
\newcommand{\ControlFlowTok}[1]{\textcolor[rgb]{0.13,0.29,0.53}{\textbf{{#1}}}}
\newcommand{\OperatorTok}[1]{\textcolor[rgb]{0.81,0.36,0.00}{\textbf{{#1}}}}
\newcommand{\BuiltInTok}[1]{{#1}}
\newcommand{\ExtensionTok}[1]{{#1}}
\newcommand{\PreprocessorTok}[1]{\textcolor[rgb]{0.56,0.35,0.01}{\textit{{#1}}}}
\newcommand{\AttributeTok}[1]{\textcolor[rgb]{0.77,0.63,0.00}{{#1}}}
\newcommand{\RegionMarkerTok}[1]{{#1}}
\newcommand{\InformationTok}[1]{\textcolor[rgb]{0.56,0.35,0.01}{\textbf{\textit{{#1}}}}}
\newcommand{\WarningTok}[1]{\textcolor[rgb]{0.56,0.35,0.01}{\textbf{\textit{{#1}}}}}
\newcommand{\AlertTok}[1]{\textcolor[rgb]{0.94,0.16,0.16}{{#1}}}
\newcommand{\ErrorTok}[1]{\textcolor[rgb]{0.64,0.00,0.00}{\textbf{{#1}}}}
\newcommand{\NormalTok}[1]{{#1}}
\usepackage{graphicx,grffile}
\makeatletter
\def\maxwidth{\ifdim\Gin@nat@width>\linewidth\linewidth\else\Gin@nat@width\fi}
\def\maxheight{\ifdim\Gin@nat@height>\textheight0.8\textheight\else\Gin@nat@height\fi}
\makeatother
% Scale images if necessary, so that they will not overflow the page
% margins by default, and it is still possible to overwrite the defaults
% using explicit options in \includegraphics[width, height, ...]{}
\setkeys{Gin}{width=\maxwidth,height=\maxheight,keepaspectratio}

% Prevent slide breaks in the middle of a paragraph:
\widowpenalties 1 10000
\raggedbottom

\AtBeginPart{
\let\insertpartnumber\relax
\let\partname\relax
\frame{\partpage}
}
\AtBeginSection{
\ifbibliography
\else
\let\insertsectionnumber\relax
\let\sectionname\relax
\frame{\sectionpage}
\fi
}
\AtBeginSubsection{
\let\insertsubsectionnumber\relax
\let\subsectionname\relax
\frame{\subsectionpage}
}

\setlength{\parindent}{0pt}
\setlength{\parskip}{6pt plus 2pt minus 1pt}
\setlength{\emergencystretch}{3em}  % prevent overfull lines
\providecommand{\tightlist}{%
\setlength{\itemsep}{0pt}\setlength{\parskip}{0pt}}
\setcounter{secnumdepth}{0}

\title{LendingClub: Predicting Loan Defaults}
\author{Oliver Mills}
\date{6 June 2019}

\begin{document}
\frame{\titlepage}

\begin{frame}{Background}

LendingClub is a peer-to-peer lending platform established in the US in
2006. Borrowers can request unsecured personal loans between \$1,000 and
\$40,000 from potential investors.

LendingClub had a strong start, raising \$1 billion in the 2014 initial
public offering but ran into some investment issues in 2016 as well as
an internal scandal regarding the CEO Renaud Laplanche.

You can download several datasets from the LendinClub website about
lenders and borrowers using the platform. I decided to see if we could
use some of the variables to predict if someone has paid back their loan
in full.

\end{frame}

\begin{frame}[fragile]{Dataset}

\begin{verbatim}
## [1] 9578   14
\end{verbatim}

So we have about 10,000 observations, 13 predictor variables and one
binary response variable `not.fully.paid'.

0 = loan has been fully paid.

1 = loan has not been fully paid.

\end{frame}

\begin{frame}{Variables}

\begin{table}[H]
\centering\begingroup\fontsize{14}{16}\selectfont

\begin{tabular}{>{\bfseries}l||l}
\hline
Variable & Description\\
\hline
credit.policy & 1 if the customer meets the credit underwriting criteria of LendingClub.com, and 0 otherwise.\\
\hline
purpose & The purpose of the loan (takes values "credit\_card", "debt\_consolidation", "educational", "major\_purchase", "small\_business", and "all\_other").\\
\hline
int.rate & The interest rate of the loan, as a proportion (a rate of 11\% would be stored as 0.11). Borrowers judged by LendingClub.com to be more risky are assigned higher interest rates.\\
\hline
installment & The monthly installments owed by the borrower if the loan is funded.\\
\hline
log.annual.inc & The natural log of the self-reported annual income of the borrower.\\
\hline
dti & The debt-to-income ratio of the borrower (amount of debt divided by annual income).\\
\hline
fico & The FICO credit score of the borrower.\\
\hline
\end{tabular}
\endgroup{}
\end{table}

\end{frame}

\begin{frame}{Variables (cont.)}

\begin{table}[H]
\centering\begingroup\fontsize{14}{16}\selectfont

\begin{tabular}{>{\bfseries}l||l}
\hline
Variable & Description\\
\hline
days.with.cr.line & The number of days the borrower has had a credit line.\\
\hline
revol.bal & The borrower's revolving balance (amount unpaid at the end of the credit card billing cycle).\\
\hline
revol.util & The borrower's revolving line utilization rate (the amount of the credit line used relative to total credit available).\\
\hline
inq.last.6mths & The borrower's number of inquiries by creditors in the last 6 months.\\
\hline
delinq.2yrs & The number of times the borrower had been 30+ days past due on a payment in the past 2 years.\\
\hline
pub.rec & The borrower's number of derogatory public records (bankruptcy filings, tax liens, or judgments).\\
\hline
\end{tabular}
\endgroup{}
\end{table}

\end{frame}

\begin{frame}{Exploration}

Lower credit scores appear to have more loans in general - the data
appears to be a positively skewed multimodal distribution (light blue =
fully paid, dark blue = not fully paid).

\includegraphics{AzzurroPresentation_files/figure-beamer/unnamed-chunk-6-1.pdf}

\end{frame}

\begin{frame}

Lots of debt consolidation loans - there doesn't appear to be much
difference between the groups in terms of loan payment status

\includegraphics{AzzurroPresentation_files/figure-beamer/unnamed-chunk-7-1.pdf}

\end{frame}

\begin{frame}

As expected, lower FICO/credit scores result in higher interest rates.
There doesn't appear to be any distinct difference in loan payment
status groups.

\includegraphics{AzzurroPresentation_files/figure-beamer/unnamed-chunk-8-1.pdf}

\end{frame}

\begin{frame}

If we look at the distribution between loan status we can see that there
are alot more people paying the full loan than those who are charged
off.

We have an imbalanced response variable meaning that I will have
upsample the minority class (`oversampling') during the train/test
split.

\includegraphics{AzzurroPresentation_files/figure-beamer/unnamed-chunk-9-1.pdf}

\end{frame}

\begin{frame}[fragile]

Let's check the completeness and variance inflation factors (VIF) of the
dataset. No missing values and VIF \textless{} 4 which means we probably
don't have multi-collinearity issues.

\includegraphics{AzzurroPresentation_files/figure-beamer/unnamed-chunk-10-1.pdf}

\begin{Shaded}
\begin{Highlighting}[]
\KeywordTok{which}\NormalTok{(multi.col$VIF>}\DecValTok{4}\NormalTok{)}
\end{Highlighting}
\end{Shaded}

\begin{verbatim}
## integer(0)
\end{verbatim}

\end{frame}

\begin{frame}[fragile]{Model 1}

Let's first try a Support Vector Machine (SVM) model. I've tried going
to try a 70/30 split for the train and test datasets.

\begin{Shaded}
\begin{Highlighting}[]
\KeywordTok{table}\NormalTok{(predicted.svm, loan_test$not.fully.paid, }\DataTypeTok{dnn =} \KeywordTok{c}\NormalTok{(}\StringTok{"Predicted"}\NormalTok{,}\StringTok{"Actual"}\NormalTok{))}
\end{Highlighting}
\end{Shaded}

\begin{verbatim}
##          Actual
## Predicted    0    1
##         0 1878  323
##         1  535  136
\end{verbatim}

The model has done well in predicting those that will pay off their
loan, but not so well for those that won't. Let's take a look at the ROC
and AUC.

\end{frame}

\begin{frame}[fragile]

\begin{Shaded}
\begin{Highlighting}[]
\KeywordTok{plot}\NormalTok{(roc.svm,}\DataTypeTok{legacy.axes=}\NormalTok{T,}\DataTypeTok{print.auc=}\NormalTok{T, }\DataTypeTok{col=}\StringTok{"red"}\NormalTok{,}\DataTypeTok{main=}\StringTok{"ROC and AUC(SVM Model)"}\NormalTok{)}
\end{Highlighting}
\end{Shaded}

\includegraphics{AzzurroPresentation_files/figure-beamer/unnamed-chunk-16-1.pdf}

The ROC curve is a performance metric for classification problems. The
higher the area under the curve (AUC) is, the better the model is at
predicting 0's as 0's and 1's as 1's. So, this model isn't very good at
discriminating between our two classes. Let's try something else.

\end{frame}

\begin{frame}[fragile]{Model 2}

Let's fit a logistic regression model. I'll use the same 70/30 split,
and create a prediction cutoff so that if the model predicts a
value\textgreater{}0.5 it becomes 1 and 0 otherwise.

\begin{Shaded}
\begin{Highlighting}[]
\KeywordTok{table}\NormalTok{(loan_test$not.fully.paid,pred_cut_off)}
\end{Highlighting}
\end{Shaded}

\begin{verbatim}
##    pred_cut_off
##        0    1
##   0 1588  825
##   1  202  257
\end{verbatim}

\end{frame}

\begin{frame}[fragile]

\begin{verbatim}
## [1] 0.6090074
\end{verbatim}

\begin{Shaded}
\begin{Highlighting}[]
\CommentTok{#ROC Curve}
\KeywordTok{roc.curve}\NormalTok{(loan_test$not.fully.paid, pred_cut_off, }\DataTypeTok{col=}\StringTok{"red"}\NormalTok{, }\DataTypeTok{main=}\StringTok{"The ROC-curve for Model with cut-off=0.5"}\NormalTok{)}
\end{Highlighting}
\end{Shaded}

\includegraphics{AzzurroPresentation_files/figure-beamer/unnamed-chunk-20-1.pdf}

\begin{verbatim}
## Area under the curve (AUC): 0.609
\end{verbatim}

\end{frame}

\begin{frame}{Summary}

The first model (SVM) had an AUC of 0.537 and the second logistic
regression model had a slightly better AUC of 0.61.

A perfect model would have an AUC of 1, whereas an AUC of 0.5 would be
similar to tossing a coin - so both models aren't fantastic\ldots{}

Although I looked online and I think the highest AUC I could find was
about 0.7 for this dataset

\end{frame}

\begin{frame}{Lessons learnt?}

\begin{enumerate}
\def\labelenumi{\arabic{enumi}.}
\tightlist
\item
  Choose an easier dataset, especially for an interview workshop :)
\item
  Don't always jump to the fancy models - I spent alot of time trying to
  tune the SVM model.
\end{enumerate}

\end{frame}

\begin{frame}{Questions?}

\end{frame}

\end{document}
